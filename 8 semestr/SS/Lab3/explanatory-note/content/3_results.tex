%%%%%%%%%%%%%%%%%%% 3
\sectionCenteredToc{Заключение}

В процессе выполнения работы нами были рассмотрены основные задачи, выполняемые
посредством Fibre Channel SAN, а именно:

\begin{itemize}
    \item определена конфигурация настроек свитча;
    \item определена принадлежность к СХД и конфигурация виртуального
          коммутатора.
\end{itemize}

Полученные знания были применены для решения задач, возникших в ходе работы:

\begin{enumerate_num}
    \item Настройки FC SAN.
    \item Установки используемое ПО.
    \item Исследования FC SAN Trace.
\end{enumerate_num}

В процессе настройки FC SAN были решены поставленные подзадачи, а именно:

\begin{itemize}
    \item определены мировые имена портов хранения;
    \item определены мировые имена портов хоста-инициатора;
    \item предложены изменения, которые необходимо внести в конфигурацию.
\end{itemize}

В процессе выполнения лабораторной работы была проведена установка недостающего
программного обеспечения.

В процессе исследования FS SAN Trace были даны исчерпывающие ответы на все
поставленные вопросы:

\begin{enumerate_num}
    \item Что такое FLOGI?
    \item Какое мировое имя у первого порта принадлежащего Fibre Channel
    Fabric?
    \item Почему поле идентификатора источника (S\_ID) кадра FLOGI содержит
    одни нули?
    \item Какой адрес назначен первому порту принадлежащему Fibre Channel
    Fabric?
    \item Какое шестнадцатеричное представление FC-4 TYPE запрашивается для
    заданного кадра и какой протокол оно представляет?
    \item Какой сервис ответственен за GID\_FT (Get Port IDs) запрос?
\end{enumerate_num}
